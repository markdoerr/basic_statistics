\documentclass[a4paper,10pt]{book}\usepackage[]{graphicx}\usepackage[]{color}
%% maxwidth is the original width if it is less than linewidth
%% otherwise use linewidth (to make sure the graphics do not exceed the margin)
\makeatletter
\def\maxwidth{ %
  \ifdim\Gin@nat@width>\linewidth
    \linewidth
  \else
    \Gin@nat@width
  \fi
}
\makeatother

\definecolor{fgcolor}{rgb}{0.345, 0.345, 0.345}
\newcommand{\hlnum}[1]{\textcolor[rgb]{0.686,0.059,0.569}{#1}}%
\newcommand{\hlstr}[1]{\textcolor[rgb]{0.192,0.494,0.8}{#1}}%
\newcommand{\hlcom}[1]{\textcolor[rgb]{0.678,0.584,0.686}{\textit{#1}}}%
\newcommand{\hlopt}[1]{\textcolor[rgb]{0,0,0}{#1}}%
\newcommand{\hlstd}[1]{\textcolor[rgb]{0.345,0.345,0.345}{#1}}%
\newcommand{\hlkwa}[1]{\textcolor[rgb]{0.161,0.373,0.58}{\textbf{#1}}}%
\newcommand{\hlkwb}[1]{\textcolor[rgb]{0.69,0.353,0.396}{#1}}%
\newcommand{\hlkwc}[1]{\textcolor[rgb]{0.333,0.667,0.333}{#1}}%
\newcommand{\hlkwd}[1]{\textcolor[rgb]{0.737,0.353,0.396}{\textbf{#1}}}%

\usepackage{framed}
\makeatletter
\newenvironment{kframe}{%
 \def\at@end@of@kframe{}%
 \ifinner\ifhmode%
  \def\at@end@of@kframe{\end{minipage}}%
  \begin{minipage}{\columnwidth}%
 \fi\fi%
 \def\FrameCommand##1{\hskip\@totalleftmargin \hskip-\fboxsep
 \colorbox{shadecolor}{##1}\hskip-\fboxsep
     % There is no \\@totalrightmargin, so:
     \hskip-\linewidth \hskip-\@totalleftmargin \hskip\columnwidth}%
 \MakeFramed {\advance\hsize-\width
   \@totalleftmargin\z@ \linewidth\hsize
   \@setminipage}}%
 {\par\unskip\endMakeFramed%
 \at@end@of@kframe}
\makeatother

\definecolor{shadecolor}{rgb}{.97, .97, .97}
\definecolor{messagecolor}{rgb}{0, 0, 0}
\definecolor{warningcolor}{rgb}{1, 0, 1}
\definecolor{errorcolor}{rgb}{1, 0, 0}
\newenvironment{knitrout}{}{} % an empty environment to be redefined in TeX

\usepackage{alltt}
\usepackage[utf8]{inputenc}
\usepackage{graphicx, subfig}

\title{Basic Statistics for Scientists in R and python}
\date{\today}
\author{mark doerr \\ institute for biochemistry \\ university greifswald, germany}
\IfFileExists{upquote.sty}{\usepackage{upquote}}{}
\begin{document}
\maketitle
\setkeys{Gin}{width=width=0.8\textwidth}

\section{Using R Studio}

\subsection{Windows in R Studio}

\begin{itemize}
 \item Text Editor window 
 \item Console
 \item Environment window
 \item Help and Plots Window
\end{itemize}

\subsection{Getting help}

Help/documentation viewer

Hitting F1 on a function shows help.

\section{Basic Data Types in R}

\subsection{vector}

\begin{knitrout}
\definecolor{shadecolor}{rgb}{0.969, 0.969, 0.969}\color{fgcolor}\begin{kframe}
\begin{alltt}
\hlstd{x} \hlkwb{=} \hlnum{1}

\hlstd{x}
\end{alltt}
\begin{verbatim}
## [1] 1
\end{verbatim}
\begin{alltt}
\hlstd{x[}\hlnum{1}\hlstd{]}
\end{alltt}
\begin{verbatim}
## [1] 1
\end{verbatim}
\begin{alltt}
\hlstd{y} \hlkwb{=} \hlstd{(}\hlnum{1}\hlopt{:}\hlnum{10}\hlstd{)}

\hlstd{y}
\end{alltt}
\begin{verbatim}
##  [1]  1  2  3  4  5  6  7  8  9 10
\end{verbatim}
\begin{alltt}
\hlstd{y[}\hlnum{2}\hlstd{]}
\end{alltt}
\begin{verbatim}
## [1] 2
\end{verbatim}
\end{kframe}
\end{knitrout}


\subsection{matrix}

\begin{knitrout}
\definecolor{shadecolor}{rgb}{0.969, 0.969, 0.969}\color{fgcolor}\begin{kframe}
\begin{alltt}
\hlstd{a_mtr} \hlkwb{=} \hlkwd{matrix}\hlstd{(y,} \hlkwc{nrow}\hlstd{=}\hlnum{2}\hlstd{)}

\hlstd{a_mtr}
\end{alltt}
\begin{verbatim}
##      [,1] [,2] [,3] [,4] [,5]
## [1,]    1    3    5    7    9
## [2,]    2    4    6    8   10
\end{verbatim}
\end{kframe}
\end{knitrout}

\subsection{list}

\begin{knitrout}
\definecolor{shadecolor}{rgb}{0.969, 0.969, 0.969}\color{fgcolor}\begin{kframe}
\begin{alltt}
\hlstd{a_lst} \hlkwb{=} \hlkwd{list}\hlstd{(}\hlstr{"A"}\hlstd{,} \hlnum{1}\hlstd{)}

\hlstd{a_lst}
\end{alltt}
\begin{verbatim}
## [[1]]
## [1] "A"
## 
## [[2]]
## [1] 1
\end{verbatim}
\end{kframe}
\end{knitrout}


\subsection{data frame}

\begin{knitrout}
\definecolor{shadecolor}{rgb}{0.969, 0.969, 0.969}\color{fgcolor}\begin{kframe}
\begin{alltt}
\hlstd{x} \hlkwb{=} \hlstd{(}\hlnum{1}\hlopt{:}\hlnum{10}\hlstd{)}
\hlstd{my_first_data_frame_df} \hlkwb{=} \hlkwd{data.frame}\hlstd{(}\hlstr{"x"}\hlstd{=x,} \hlstr{"y"}\hlstd{=x}\hlopt{*}\hlnum{0.1} \hlstd{)}

\hlstd{my_first_data_frame_df}
\end{alltt}
\begin{verbatim}
##     x   y
## 1   1 0.1
## 2   2 0.2
## 3   3 0.3
## 4   4 0.4
## 5   5 0.5
## 6   6 0.6
## 7   7 0.7
## 8   8 0.8
## 9   9 0.9
## 10 10 1.0
\end{verbatim}
\end{kframe}
\end{knitrout}

This shows how to access the data of the data frame

\begin{knitrout}
\definecolor{shadecolor}{rgb}{0.969, 0.969, 0.969}\color{fgcolor}\begin{kframe}
\begin{alltt}
\hlstd{my_first_data_frame_df[,}\hlnum{1}\hlstd{]} \hlcom{# first column}
\end{alltt}
\begin{verbatim}
##  [1]  1  2  3  4  5  6  7  8  9 10
\end{verbatim}
\begin{alltt}
\hlstd{my_first_data_frame_df}\hlopt{$}\hlstd{x}  \hlcom{# first column by name}
\end{alltt}
\begin{verbatim}
##  [1]  1  2  3  4  5  6  7  8  9 10
\end{verbatim}
\begin{alltt}
\hlstd{my_first_data_frame_df[}\hlnum{1}\hlstd{,]} \hlcom{# first line}
\end{alltt}
\begin{verbatim}
##   x   y
## 1 1 0.1
\end{verbatim}
\begin{alltt}
\hlstd{my_first_data_frame_df[}\hlnum{2}\hlstd{,}\hlnum{2}\hlstd{]} \hlcom{# second element of second line}
\end{alltt}
\begin{verbatim}
## [1] 0.2
\end{verbatim}
\end{kframe}
\end{knitrout}



\begin{knitrout}
\definecolor{shadecolor}{rgb}{0.969, 0.969, 0.969}\color{fgcolor}\begin{kframe}
\begin{alltt}
\hlstd{x} \hlkwb{<-} \hlnum{1}\hlopt{:}\hlnum{10}
\hlstd{w} \hlkwb{<-} \hlnum{20} \hlopt{+} \hlnum{10}\hlopt{*}\hlstd{x}
\hlstd{w}
\end{alltt}
\begin{verbatim}
##  [1]  30  40  50  60  70  80  90 100 110 120
\end{verbatim}
\begin{alltt}
\hlstd{linear_sample_df} \hlkwb{<-} \hlkwd{data.frame}\hlstd{(}\hlkwc{x}\hlstd{=x,} \hlkwc{y}\hlstd{=w} \hlopt{+} \hlkwd{rnorm}\hlstd{(}\hlnum{10}\hlstd{)}\hlopt{*}\hlnum{10}\hlstd{)}

\hlkwd{plot}\hlstd{(linear_sample_df)}

\hlstd{linear_model_lm} \hlkwb{<-} \hlkwd{lm}\hlstd{(y} \hlopt{~} \hlstd{x,} \hlkwc{data}\hlstd{=linear_sample_df)}
\hlkwd{summary}\hlstd{(linear_model_lm)}
\end{alltt}
\begin{verbatim}
## 
## Call:
## lm(formula = y ~ x, data = linear_sample_df)
## 
## Residuals:
##      Min       1Q   Median       3Q      Max 
## -10.4428  -4.8458  -0.8379   4.9563   9.3348 
## 
## Coefficients:
##             Estimate Std. Error t value Pr(>|t|)    
## (Intercept)  28.1239     4.8138   5.842 0.000386 ***
## x             8.7173     0.7758  11.236 3.53e-06 ***
## ---
## Signif. codes:  0 '***' 0.001 '**' 0.01 '*' 0.05 '.' 0.1 ' ' 1
## 
## Residual standard error: 7.047 on 8 degrees of freedom
## Multiple R-squared:  0.9404,	Adjusted R-squared:  0.933 
## F-statistic: 126.3 on 1 and 8 DF,  p-value: 3.533e-06
\end{verbatim}
\begin{alltt}
\hlkwd{abline}\hlstd{(linear_model_lm,} \hlkwc{col}\hlstd{=}\hlstr{"red"}\hlstd{)}
\end{alltt}
\end{kframe}
\includegraphics[width=\maxwidth]{figure/in-build_data_sets_-_data_frame_2-1} 

\end{knitrout}


\subsection{Examples of in-build data sets for testing}

\begin{knitrout}
\definecolor{shadecolor}{rgb}{0.969, 0.969, 0.969}\color{fgcolor}\begin{kframe}
\begin{alltt}
\hlkwd{library}\hlstd{(}\hlkwc{help} \hlstd{=} \hlstr{"datasets"}\hlstd{)}
\end{alltt}
\end{kframe}
\end{knitrout}

\subsubsection{Iris}

\begin{knitrout}
\definecolor{shadecolor}{rgb}{0.969, 0.969, 0.969}\color{fgcolor}\begin{kframe}
\begin{alltt}
\hlkwd{head}\hlstd{(iris)}
\end{alltt}
\begin{verbatim}
##   Sepal.Length Sepal.Width Petal.Length Petal.Width Species
## 1          5.1         3.5          1.4         0.2  setosa
## 2          4.9         3.0          1.4         0.2  setosa
## 3          4.7         3.2          1.3         0.2  setosa
## 4          4.6         3.1          1.5         0.2  setosa
## 5          5.0         3.6          1.4         0.2  setosa
## 6          5.4         3.9          1.7         0.4  setosa
\end{verbatim}
\begin{alltt}
\hlkwd{head}\hlstd{(iris3)}
\end{alltt}
\begin{verbatim}
## [1] 5.1 4.9 4.7 4.6 5.0 5.4
\end{verbatim}
\end{kframe}
\end{knitrout}

\subsubsection*{women}

\begin{knitrout}
\definecolor{shadecolor}{rgb}{0.969, 0.969, 0.969}\color{fgcolor}\begin{kframe}
\begin{verbatim}
##   height weight
## 1     58    115
## 2     59    117
## 3     60    120
## 4     61    123
## 5     62    126
## 6     63    129
\end{verbatim}
\end{kframe}
\end{knitrout}

\subsubsection*{ELISA - DNAse}

\begin{knitrout}
\definecolor{shadecolor}{rgb}{0.969, 0.969, 0.969}\color{fgcolor}\begin{kframe}
\begin{verbatim}
##   Run       conc density
## 1   1 0.04882812   0.017
## 2   1 0.04882812   0.018
## 3   1 0.19531250   0.121
## 4   1 0.19531250   0.124
## 5   1 0.39062500   0.206
## 6   1 0.39062500   0.215
\end{verbatim}
\end{kframe}
\end{knitrout}


\section*{Mean, Average, Summary}

\begin{knitrout}
\definecolor{shadecolor}{rgb}{0.969, 0.969, 0.969}\color{fgcolor}\begin{kframe}
\begin{verbatim}
## [1] 60.1
## [1] 4.998889
##    Min. 1st Qu.  Median    Mean 3rd Qu.    Max. 
##    51.0    57.5    61.0    60.1    62.5    69.0
## [1] 61 59 55
## 
## 	One Sample t-test
## 
## data:  wtcsf[1:4]
## t = 31.6563, df = 3, p-value = 6.927e-05
## alternative hypothesis: true mean is not equal to 0
## 95 percent confidence interval:
##  53.74326 65.75674
## sample estimates:
## mean of x 
##     59.75
## 
## 	One Sample t-test
## 
## data:  wtcsf
## t = 38.019, df = 9, p-value = 2.991e-11
## alternative hypothesis: true mean is not equal to 0
## 95 percent confidence interval:
##  56.52401 63.67599
## sample estimates:
## mean of x 
##      60.1
## [1] 45.5
## [1] 7.382412
##    Min. 1st Qu.  Median    Mean 3rd Qu.    Max. 
##   34.00   40.25   46.50   45.50   49.50   59.00
\end{verbatim}
\end{kframe}
\end{knitrout}

\section*{Reading/Writing Data from a File}



\section*{Basic Plotting in R}

\begin{knitrout}
\definecolor{shadecolor}{rgb}{0.969, 0.969, 0.969}\color{fgcolor}\begin{kframe}
\begin{alltt}
\hlstd{x} \hlkwb{<-} \hlkwd{rnorm}\hlstd{(}\hlnum{10}\hlstd{);}

\hlkwd{plot}\hlstd{(x)}
\end{alltt}
\end{kframe}
\includegraphics[width=\maxwidth]{figure/basic_plotting-1} 
\begin{kframe}\begin{alltt}
\hlkwd{plot}\hlstd{(x,} \hlkwc{type}\hlstd{=}\hlstr{"l"}\hlstd{,} \hlkwc{col}\hlstd{=}\hlstr{"red"}\hlstd{,} \hlkwc{main}\hlstd{=}\hlstr{"Line Diagramm"}\hlstd{,} \hlkwc{xlab}\hlstd{=}\hlstr{"time [year]"}\hlstd{,} \hlkwc{ylab}\hlstd{=}\hlstr{"difference/year"}\hlstd{)}
\end{alltt}
\end{kframe}
\includegraphics[width=\maxwidth]{figure/basic_plotting-2} 
\begin{kframe}\begin{alltt}
\hlstd{x} \hlkwb{<-} \hlstd{(}\hlnum{0}\hlopt{:}\hlnum{10}\hlstd{)}
\hlstd{y} \hlkwb{<-} \hlkwd{sin}\hlstd{(x)}

\hlstd{x}
\end{alltt}
\begin{verbatim}
##  [1]  0  1  2  3  4  5  6  7  8  9 10
\end{verbatim}
\begin{alltt}
\hlstd{y}
\end{alltt}
\begin{verbatim}
##  [1]  0.0000000  0.8414710  0.9092974  0.1411200 -0.7568025 -0.9589243
##  [7] -0.2794155  0.6569866  0.9893582  0.4121185 -0.5440211
\end{verbatim}
\begin{alltt}
\hlkwd{plot}\hlstd{(x, y)}
\end{alltt}
\end{kframe}
\includegraphics[width=\maxwidth]{figure/basic_plotting-3} 
\begin{kframe}\begin{alltt}
\hlkwd{plot}\hlstd{(x, y,} \hlkwc{type}\hlstd{=}\hlstr{"l"}\hlstd{)}
\end{alltt}
\end{kframe}
\includegraphics[width=\maxwidth]{figure/basic_plotting-4} 
\begin{kframe}\begin{alltt}
\hlstd{x} \hlkwb{<-} \hlkwd{rnorm}\hlstd{(}\hlnum{10}\hlstd{); y} \hlkwb{<-} \hlkwd{rnorm}\hlstd{(}\hlnum{10}\hlstd{)}
\hlkwd{plot}\hlstd{(x,y)}
\end{alltt}
\end{kframe}
\includegraphics[width=\maxwidth]{figure/basic_plotting-5} 
\begin{kframe}\begin{alltt}
\hlstd{x} \hlkwb{<-} \hlnum{1}\hlopt{:}\hlnum{10}
\hlstd{w} \hlkwb{<-} \hlnum{20} \hlopt{+} \hlnum{10}\hlopt{*}\hlstd{x}
\hlstd{w}
\end{alltt}
\begin{verbatim}
##  [1]  30  40  50  60  70  80  90 100 110 120
\end{verbatim}
\begin{alltt}
\hlstd{linear_sample_df} \hlkwb{<-} \hlkwd{data.frame}\hlstd{(}\hlkwc{x}\hlstd{=x,} \hlkwc{y}\hlstd{=w} \hlopt{+} \hlkwd{rnorm}\hlstd{(}\hlnum{10}\hlstd{)}\hlopt{*}\hlnum{10}\hlstd{)}

\hlkwd{plot}\hlstd{(linear_sample_df)}

\hlstd{linear_model_lm} \hlkwb{<-} \hlkwd{lm}\hlstd{(y} \hlopt{~} \hlstd{x,} \hlkwc{data}\hlstd{=linear_sample_df)}
\hlkwd{summary}\hlstd{(linear_model_lm)}
\end{alltt}
\begin{verbatim}
## 
## Call:
## lm(formula = y ~ x, data = linear_sample_df)
## 
## Residuals:
##      Min       1Q   Median       3Q      Max 
## -10.9983  -3.6736  -0.8972   6.7122  10.6759 
## 
## Coefficients:
##             Estimate Std. Error t value Pr(>|t|)    
## (Intercept)  26.3427     5.5620   4.736  0.00147 ** 
## x             8.8894     0.8964   9.917 9.03e-06 ***
## ---
## Signif. codes:  0 '***' 0.001 '**' 0.01 '*' 0.05 '.' 0.1 ' ' 1
## 
## Residual standard error: 8.142 on 8 degrees of freedom
## Multiple R-squared:  0.9248,	Adjusted R-squared:  0.9154 
## F-statistic: 98.34 on 1 and 8 DF,  p-value: 9.034e-06
\end{verbatim}
\begin{alltt}
\hlkwd{abline}\hlstd{(linear_model_lm,} \hlkwc{col}\hlstd{=}\hlstr{"red"}\hlstd{)}
\end{alltt}
\end{kframe}
\includegraphics[width=\maxwidth]{figure/basic_plotting-6} 

\end{knitrout}


\section*{Variance Tests}




\section*{Significance Tests}




\section*{Distributions}




\subsection*{Normal Distribution}

Normal Distribution 

Based on the equation $$ f(x) = e^{-\frac{(x-\mu)^2}{2\sigma^2}} $$

$\mu = $ mean
$\sigma = $ standard deviation

with $ mean = 1 $, $ \sigma = 0.1 $

\begin{figure}
\begin{center}
\begin{knitrout}
\definecolor{shadecolor}{rgb}{0.969, 0.969, 0.969}\color{fgcolor}
\includegraphics[width=\maxwidth]{figure/normal_distribution_example1-1} 

\end{knitrout}
\caption{Normal Distribution.}
\end{center}
\end{figure}

\subsubsection*{Example of a Normal Distribution}

Children's IQ scores are normally distributed with a
mean of 100 and a standard deviation of 15. What
proportion of children are expected to have an IQ between
80 and 120?

\begin{knitrout}
\definecolor{shadecolor}{rgb}{0.969, 0.969, 0.969}\color{fgcolor}
\includegraphics[width=\maxwidth]{figure/normal_distribution-1} 

\end{knitrout}


Cusum Example 

\begin{knitrout}
\definecolor{shadecolor}{rgb}{0.969, 0.969, 0.969}\color{fgcolor}\begin{kframe}
\begin{verbatim}
## [1] -1.939758
\end{verbatim}
\end{kframe}
\includegraphics[width=\maxwidth]{figure/cumsum_example-1} 

\end{knitrout}

\subsection*{Student Distribution}

Display the Student's t distributions with various
degrees of freedom and compare to the normal distribution

\begin{knitrout}
\definecolor{shadecolor}{rgb}{0.969, 0.969, 0.969}\color{fgcolor}
\includegraphics[width=\maxwidth]{figure/student_distribution-1} 

\end{knitrout}


\end{document}
